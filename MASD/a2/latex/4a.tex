This task has been completed in the Jupyter Notebook, which has been handed in separately. Code and the images generated can be seen in the appendix.
\subsection{}
This task has been completed in the Jupyter Notebook, which has been handed in separately. Code and the images generated can be seen in the appendix. When plotted side by side, it can be seen that the different partial derivatives are more sensitive to changes in their own direction, x and y respectively.
\subsection{}
Increasing the step length, $h$ seems to make edges stand out more, making them more readable to humans. However, this is only at small step lengths, at larger lengths the image seems to "split in two", meaning that it looks to the human eye as if there are two versions of the image, one much darker, the other much brighter.
\subsection{}
This task has been completed in the Jupyter Notebook, which has been handed in separately. Code and the images generated can be seen in the appendix.
\subsection{}
The gradient magnitude might be useful for making computers analyse different parts of images, and processing this data in an efficient way.