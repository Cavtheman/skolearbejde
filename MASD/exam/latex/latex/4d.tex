To show that $cov(V,W) = 0 \implies$ $V$ and $W$ are independent for Bernoulli distributions of $V$ and $W$, we first have to look at the definition of independence, as per Definition 1.15 in "MASD\_essentials\_ch1-updated.pdf" on Absalon. I will rewrite it slightly so it fits the exercise at hand:\\
$$
p(x,y)=p(x)p(y) \text{ for all }x,y \implies \text{mutual independence}
$$
So now we can see that we need to show that:
$$
\mathbb{E}VW-\mathbb{E}V \mathbb{E}W = 0 \implies p(v,w)-p(v)p(w) = 0 \implies p(v,w) = p(v)p(w)
$$
We know that for Bernoulli distributions $\mathbb{E}V = \theta$ since:
$$
\mathbb{E}V = \sum_v vp(v) = \mathbb{P}(V=0)*0+\mathbb{P}(V=1)*1 = \theta*1+(1-\theta)*0 = \theta
$$
The same of course holds for $\mathbb{E}W = \phi$\\
We can do the same for $\mathbb{E}VW$:
$$
\mathbb{E}VW = \sum_{v,w} vwp(v,w)
$$
Since this expression always will be zero except in $(V=1,W=1)$, we can simplify the equation:
$$
\mathbb{E}VW = \theta \phi
$$
Now we can insert these values into $p(v),p(w)$ and $p(v,w)$:
$$
p(v) =
\left\{
\begin{array}{ll}
1 - \mathbb{E}V & \text{if } v=0\\
\mathbb{E}V     & \text{if } v=1
\end{array}
\right.
$$
$$
p(w) =
\left\{
\begin{array}{ll}
1 - \mathbb{E}W & \text{if } w=0\\
\mathbb{E}W     & \text{if } w=1
\end{array}
\right.
$$
$$
p(v,w) =
\left\{
\begin{array}{ll}
(1 - \mathbb{E}V)*(1-\mathbb{E}W) & \text{if } v=0,w=0\\
\mathbb{E}V*(1-\mathbb{E}W)       & \text{if } v=1,w=0\\
(1-\mathbb{E}V)*\mathbb{E}W       & \text{if } v=0,w=1\\
\mathbb{E}V \mathbb{E}W           & \text{if } v=1,w=1\\
\end{array}
\right.
$$
We can now see that for all values of $V$ and $W$, the following must be true:
$$
\mathbb{E}VW-\mathbb{E}V \mathbb{E}W = 0 \implies p(v,w)-p(v)p(w) = 0 \implies p(v,w) = p(v)p(w)
$$
And since $cos(V,W)= \mathbb{E}VW-\mathbb{E}V \mathbb{E}W$, the proof is finished.