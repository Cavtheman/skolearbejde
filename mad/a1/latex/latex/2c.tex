This part of the exercise has been completed in the attached Jupyter Notebook named "\textit{A1.ipynb}".\\
The relevant code can be seen here:
\begin{verbatim}
def estimator(samples):
    # Input is an series of vectors
    successes = 0
    for i in range (len(samples[0])):
        # Calculating the value for all given vectors,
        # and checking whether they meet the given criteria
        if ((samples[0][i])**2 + (samples[1][i])**2 <= 1):
            successes+=1
    # Normalising the end result to get an estimate of how
    # likely it is for the statement to be true
    return successes/len(samples[0])
\end{verbatim}
The estimator is given a range of samples and calculates the average value from it, with the criteria:
$$
\mathbb{P}(X^2+Y^2 \leq 1)
$$
The resulting values look as follows:
\begin{itemize}
\item estimator at sample size 1000: 0.766
\item estimator at sample size 10000: 0.7893
\item estimator at sample size 100000: 0.78474
\item estimator at sample size 1000000: 0.785553
\end{itemize}
We can see that this approaches the value $\pi/4 = 0.78539$ as the samples sizes grow larger.