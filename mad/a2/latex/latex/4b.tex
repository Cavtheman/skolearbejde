To perform the t-test, we have to follow the six steps. For this we will define the sequence $((X_1 - Y_1), \dotsc , (X_n - Y_n))$\\
First, we write the model. This is given as a normal distribution:
$$
T  \sim \mathcal{N} (\mu,\sigma^2)
$$
Then we write down the hypothesis. This hypothesis must be the case since if there is no difference between the distributions, they will have the same mean.
$$
H_0 : \mu = 0, H_1 : \mu \neq 0
$$
Now we write down the test:
$$
T = \sqrt{n}\frac{\bar X - \mu}{\sigma}
$$
We are given the significance level $\alpha = 0.05$
We calculate the rejection region to be $\mathcal{R} = (-\infty,z_{0.05}]\cup [z_{0.95}, \infty) = (-\infty,-2.776]\cup [2.776, \infty)$
Finally we compute the test statistic with the data, using python. We have done this using python in the attached Jupyter Notebook:
$$
T = \sqrt{n}\frac{\bar X - \mu}{\sigma} = 1.809
$$
This is not $\in \mathcal{R}$, which means the test passes.