First, we will find the likelihood, which is the same as the $joint \text{ }pdf$, which is defined as thus:
$$
f_{\theta}^{joint}(x_1,...,x_n) = \prod_{i = 1}^n f_\theta(x_n)
$$
We can then figure out that the likelihood is:
$$
f_{\beta}^{joint}(x_1,...,x_n) = 
\left\{
\begin{matrix}
\prod_{i = 1}^n f_\beta(x_n) & \text{if }0 \leq x \leq \beta \\
0 & otherwise.
\end{matrix}
\right.
=
\left\{
\begin{matrix}
\prod_{i = 1}^n \frac{2}{\beta^2} \cdot (\beta - x_n) & \text{if }0 \leq x \leq \beta \\
0 & otherwise.
\end{matrix}
\right.
$$
We then have to figure out the maximum likelihood estimator. This is done using the following formula:
$$
\hat\theta_n^{ML}(x_1,...,x_n) = \underset{\theta}{argmax}\prod_{i=1}^n f_\theta(x_n)
$$
We insert the previous formula with the given values for $X$:
$$
\hat\beta_n^{ML}(x_1,...,x_n) = \underset{\beta}{argmax}\prod_{i=1}^n f_\beta(x_n) = \underset{\beta}{argmax}\Bigg(\frac{2}{\beta^2} \cdot (\beta - 3) \cdot \frac{2}{\beta^2} \cdot (\beta - 4)\Bigg)
$$
First we rewrite it to make our lives easier:
$$
\frac{4(\beta-4)(\beta-3)}{\beta^4}
$$
Now we have to differentiate it to find the local maxima:
$$
\frac{d}{d\beta} \frac{4(\beta-4)(\beta-3)}{\beta^4} = \frac{-4(2\beta^2-21\beta+48)}{\beta^5}
$$
Then we have to set this to $0$:
$$
\frac{-4(2\beta^2-21\beta+48)}{\beta^5} = 0
$$
And find that there are two solutions:
$$
\beta_1 = \frac{21}{4}-\frac{\sqrt{57}}{4}
$$
$$
\beta_2 = \frac{21}{4}+\frac{\sqrt{57}}{4}
$$
Then, to find whether they are maxima or minima, we compute derive again, and solve with $\beta = \beta_1$ and $\beta = \beta_2$. With this, we find that the $\underset{\beta}{argmax}$ is at:
$$
\beta = \frac{21}{4}+\frac{\sqrt{57}}{4}
$$