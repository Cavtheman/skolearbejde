To perform this statistical test, we have to go through the six steps:
\begin{itemize}
\item A model: \\
The model is given as $X \sim Bin(n,\theta)$, with $n = 20$
\item A hypothesis: \\
A hypothesis is given as well: $H_0: \theta = 0.5, H_1: \theta \neq 0.5$
\item A test statistic and distribution: \\
We use a binomial test to find the probability that a $\theta = 0.5$ is correct when $n = 20$ and $X = 13$. What this means is that we can simply say that our test statistic is:
$$
T = X
$$
This is because the rejection region we will calculate later only needs to know our $X$ value.
\item A significance level: \\
The significance level is also given as $0.05$
\item A rejection region: \\
A rejection region is given as:
$$
\mathcal{R}=\{0,...,\alpha\}\cup\{20-\alpha,...,20\}
$$
So we have to find an $\alpha$. We do this by solving the following:
$$
\underset{a}{argmax} \Bigg(\sum_{k=0}^a\Bigg(\begin{pmatrix} n \\ k \end{pmatrix} \theta^k (1-\theta)^{n-k}\Bigg) \leq 0.025 \Bigg) = 5
$$
We thus find that $\mathcal{R}=\{0,...,5\}\cup\{15,...,20\}$
\item Compute test statistic: \\
We insert the values into the test statistic to see whether the given data is within the rejection region:
$$
T = X = 13
$$
We can see that this is not within the rejection region:
$$
T = 13 \not\in \mathcal{R} = \{0,...,5\} \cup \{15,...,20\}
$$
Thus, we can attribute the higher than average number of correct guesses to chance, not the predictions.
\end{itemize}