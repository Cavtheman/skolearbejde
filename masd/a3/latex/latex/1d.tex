For this exercise we have to compute the derivatives both with respect to $x_1$ and $x_2$.
$$f(x_1,x_2) = (x_1-x_2)^2$$
$$\frac{\partial y}{\partial x_1} = 2x_1-2x_2$$
$$\frac{\partial y}{\partial x_2} = -2x_1+2x_2$$
Then we solve for $x_1$ and $x_2$, using matrices.
$$
\left[
\begin{array}{rr|r}
2 & -2 & 0\\
-2 & 2 & 0
\end{array}
\right]
=
\left[
\begin{array}{rr|r}
1 & -1 & 0\\
0 & 0 & 0
\end{array}
\right]
$$
With this we can see that $x_1-x_2=0$, in other words, $f(x_1,x_2) = 0$ when $x_1 = x_2$. This means that $f(x_1,x_2)$ intersects the x-axis at all points where $x_1 = x_2$.\\
Now, to find whether it is a local maximum or minimum, we use the eigenvalues of the Hessian matrix.
$$\frac{\partial^2 y}{\partial x_1 \partial x_2} = -2$$
$$\frac{\partial^2 y}{\partial^2 x_1} = 2$$
$$\frac{\partial^2 y}{\partial x_2 \partial x_1} = -2$$
$$\frac{\partial^2 y}{\partial^2 x_2} = 2$$
Thus, we can see that the Hessian matrix looks as follows:
$$
\left[
\begin{array}{rr}
2 & -2\\
-2 & 2
\end{array}
\right]
$$
$$
\left[
\begin{array}{rr}
2 & -2\\
-2 & 2
\end{array}
\right]
-\lambda I
=
\left[
\begin{array}{rr}
2-\lambda & -2\\
-2 & 2-\lambda
\end{array}
\right]
$$
Then we find the determinants of the matrix:
$$
D
\left[
\begin{array}{rr}
2-\lambda & -2\\
-2 & 2-\lambda
\end{array}
\right]
= \lambda^2-4\lambda
$$
Solving $\lambda^2-4\lambda = 0$ with Wolfram Alpha shows us that $\lambda_1 = 0, \lambda_2 = 4$. Since both values aren't nonzero, the test is inconclusive, and we can't tell whether it is a local maximum or minimum.