Here, we have to show that the following is true:
$$||I\odot(M-AB)||^2 = \sum_{i=1}^6\sum_{j=1}^{10} I_{ij}(M_{ij}-(a_{i1}b_{1j}+a_{i2}b_{2j}))^2$$
First we start off by showing that each element in the matrix can be written as follows:
$$AB_{ij} =(a_{i1}b_{1j}+a_{i2}b_{2j})$$
First, we have to remember the rules for matrix multiplication, and recall that any element in the resulting square matrix with length $m$ can be described as thus, when the matrices being multiplied have the forms $m\times n$ and $n \times m$.
$$a_{i1}b_{1j} + a_{i2}b_{2j} + ... + a_{in}b_{nj}$$
This can also be applied to AB, in which case we find that any element in the resulting matrix can be described as $(a_{i1}b_{1j}+a_{i2}b_{2j})$, which is exactly what we want to show.\\
\\
If we recall the rules for matrix subtraction, we can also figure out that each element in the new matrix will be defined as thus, where $X$ and $Y$ are the matrices being subtracted from one another, with the resulting matrix $Z$:
$$Z_{ij} = X_{ij} - Y_{ij}$$
This must also be true for $M-AB$:
$$M_{ij}-(a_{i1}b_{1j}+a_{i2}b_{2j})$$
Since our previous operations have all been element-wise, multiplying the matrix $I$ on element-wise is simple:
$$I_{ij}(M_{ij}-(a_{i1}b_{1j}+a_{i2}b_{2j}))$$
Now, the matrix norm is defined as being the same as one long vector $V$ with all the same elements as the matrix:
$$||V||^2 = \sum_{i=1}^{n}(V_i^2)$$
Since a matrix can be described as a vector of vectors, the following must be true for the matrix norm, of a matrix $N$, with the form $n\times m$:
$$||N||^2 = \sum_{i=1}^n\sum_{j=1}^mN_{ij}$$
With this knowledge we can conclude that the matrix norm squared of $I_{ij}(M_{ij}-(a_{i1}b_{1j}+a_{i2}b_{2j}))$ must be equal to:
$$\sum_{i=1}^6\sum_{j=1}^{10} I_{ij}(M_{ij}-(a_{i1}b_{1j}+a_{i2}b_{2j}))^2$$
Thus, we have shown that: $$||I\odot(M-AB)||^2 = \sum_{i=1}^6\sum_{j=1}^{10} I_{ij}(M_{ij}-(a_{i1}b_{1j}+a_{i2}b_{2j}))^2$$	