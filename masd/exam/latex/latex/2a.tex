\subsubsection{i)}
$$
\int_1^4 x^3-4x^2+\sqrt{x}dx
$$
First we use the sum rule:
$$
\int x^3dx-\int 4x^2dx+\int\sqrt{x}dx
$$
These are trivial, so we calculate them separately and get the following:
$$
\frac{x^4}{4}-\frac{4x^3}{3}+\frac{2}{3}x^{\frac{3}{2}}
$$
Now we calculate these at the boundaries and subtract the result of the lower from the upper:
$$
\bigg(\frac{4^4}{4}-\frac{4*4^3}{3}+\frac{2}{3}*4^\frac{3}{2}\bigg)-\bigg(\frac{1}{4}-\frac{4}{3}+\frac{2}{3}*1^\frac{3}{2}\bigg) = -16-\bigg(\frac{-5}{12}\bigg)
$$
Thus, we have found that the definite integral equals:
$$
-16-\bigg(\frac{-5}{12}\bigg)
$$
\subsubsection{ii)}
$$
\int ln(\sqrt{x})dx
$$
First we rewrite this to the following:
$$
\int \frac{ln(x)}{2}dx
$$
We now take the constant out:
$$
\frac{1}{2} \int ln(x)
$$
We now do integration by parts:
$$
\frac{1}{2}\bigg(xln(x)-\int 1dx\bigg)
$$
The integral of a constant is $\int c = x$:
$$
\frac{1}{2}\bigg(xln(x)-x\bigg)
$$
Now we have to add a constant $C$, and we're done:
$$
\frac{1}{2}\bigg(xln(x)-x\bigg)+C
$$
\subsubsection{iii)}
$$
\int x^2e^{x^3}dx
$$
Here, we have to do u-substitution:
$$
\int x^2e^udx
$$
We now have:
$$
du = 3x^2dx 
$$
$$
du*\frac{1}{3x^2} = dx
$$
With this we can substitute dx:
$$
\int e^ux^2\frac{1}{3x^2}du
$$
This can be rewritten;
$$
\int \frac{1}{3} e^udu
$$
We take the constant out, and know that $\int e^x = e^x$, so we now have the result:
$$
\frac{1}{3}\int e^{x^3}dx
$$
We end up with:
$$
\frac{1}{3}e^{x^3}
$$
\subsubsection{iv)}
$$
\int e^{-5r}dr
$$
Here, we have to do substitution again:
$$
\int e^udx
$$
We know that:
$$
du = -5dx
$$
We can then replace $dx$:
$$
\int \frac{e^u}{-5}du
$$
Now we have to take the constant out:
$$
\frac{-1}{5}\int e^udu
$$
And following the same rule as before $\int e^x = e^x$:
$$
\frac{-1}{5}e^{-5r}
$$