To show that $cov(V,W) = 0 \implies$ $V$ and $W$ are independent for Bernoulli distributions of $V$ and $W$, we first have to look at the definition of independence, as per Definition 1.15 in "MASD\_essentials\_ch1-updated.pdf" on Absalon. I will rewrite it slightly so it fits the exercise at hand:\\

\begin{align*}
\mathbb{P}(X \in \{0,1\})\mathbb{P}(Y\in \{0,1\}) =& \\
\mathbb{P}(X\in \{0,1\},Y\in \{0,1\})&
\text{ for all }x,y \implies \text{mutual independence}
\end{align*}
So now we can see that we need to show that:
\begin{align*}
&\mathbb{E}VW-\mathbb{E}V \mathbb{E}W = 0 \implies \\
&\mathbb{P}(V\in \{0,1\},W\in \{0,1\})-\mathbb{P}(V \in \{0,1\})\mathbb{P}(W\in \{0,1\}) = 0 \implies \\
&\mathbb{P}(V\in \{0,1\})\mathbb{P}(W\in \{0,1\}) = \mathbb{P}(V\in \{0,1\},W\in \{0,1\})
\end{align*}
We know that for Bernoulli distributions $\mathbb{E}V = \theta$ since:
$$
\mathbb{E}V = \sum_v vp(v) = \mathbb{P}(V=0)*0+\mathbb{P}(V=1)*1 = \theta = \mathbb{P}(V\in \{0,1\})
$$
The same holds for $\mathbb{E}W = \phi$:
$$
\mathbb{E}W = \sum_w wp(w) = \mathbb{P}(W=0)*0+\mathbb{P}(W=1)*1 = \theta = \mathbb{P}(W\in \{0,1\})
$$
To find $\mathbb{E}VW$, we have to write up the joint pmf. For brevity we will define $\mathbb{P}(V = 1) = a$, $\mathbb{P}(W = 1) =b$ and $\mathbb{P}(V = 1,W=1) = c$. We can now say the following:
$$
\mathbb{P}(V=1,W=0) = a-c \in [0,1]
$$
$$
\mathbb{P}(V=0,W=1) = b-c \in [0,1]
$$
$$
\mathbb{P}(V=0,W=0) = 1 - a-b-c \in [0,1]
$$
We can now rewrite the pmf slightly to show that $\mathbb{P}(VW \in \{0,1\}) = \mathbb{P}(V \in \{0,1\})\mathbb{P}(W \in \{0,1\})$
$$
p(v,w)=
\left\{
\begin{array}{llll}
\mathbb{P}(V=0,W=0) & = 1 - a - b - ab & = (1-a)(1-b) & = \mathbb{P}(V=0)*\mathbb{P}(W=0) \\
\mathbb{P}(V=0,W=1) & = b - ab & = (1-a)b & = \mathbb{P}(V=0)*\mathbb{P}(W=1) \\
\mathbb{P}(V=1,W=0) & = a - ba & = a(1-b) & = \mathbb{P}(V=1)*\mathbb{P}(W=0) \\
\mathbb{P}(V=1,W=1) & = ab & = ab & = \mathbb{P}(V=1)*\mathbb{P}(W=1)
\end{array}
\right.
$$
%
%p(v) =
%\left\{
%\begin{array}{ll}
%1 - \theta & \text{if } v=0\\
%\theta     & \text{if } v=1
%\end{array}
%\right.
%$$
%$$
%p(w) =
%\left\{
%\begin{array}{ll}
%1 - \phi & \text{if } w=0\\
%\phi     & \text{if } w=1
%\end{array}
%\right.
%
We can now see that for all values of $V$ and $W$, the following must be true:
$$
\mathbb{E}VW-\mathbb{E}V \mathbb{E}W = 0 \implies p(v,w)-p(v)p(w) = 0 \implies p(v,w) = p(v)p(w)
$$
And since $cos(V,W)= \mathbb{E}VW-\mathbb{E}V \mathbb{E}W$, the proof is finished.