In this exercise we have to show that for $n = 1$ and $n = 2$ with given solutions $x_1 = 1$ and $x_2 = (1-\sqrt{3})$, the values $\alpha = 0$ and $\beta = (1-\sqrt{3})^{-1}$ hold true. We do this by substituting the values in the equation:
\begin{align*}
  x_n &= \alpha(1+\sqrt{3})^n + \beta(1-\sqrt{3})^n \\
  x_1 &= 0(1+\sqrt{3})^1 + (1-\sqrt{3})^{-1}(1-\sqrt{3})^1 = \frac{(1-\sqrt{3})}{(1-\sqrt{3})} = 1 \\
  x_2 &= 0(1+\sqrt{3})^2 + (1-\sqrt{3})^{-1}(1-\sqrt{3})^2 = \frac{(1-\sqrt{3})^2}{1-\sqrt{3}} = (1-\sqrt{3})\\
\end{align*}
