\iffalse
The condition number can be found by dividing $\lambda_{max}(A)$ with $\lambda_{min}(A)$.
Using the implemented functions, we get the following values:
\begin{align*}
  \lambda_{max}(A) &= 2 \\
  \lambda_{min}(A) &= 2000.5
\end{align*}
With this we get a condition number of $0.00099975006$.
I am not quite sure how to answer the rest of the question, as I don't quite understand what it is asking.
\fi

To find the condition number, we use the following formula:
$$
_\kappa(A) = \frac{|\text{max e.v of }A|}{|\text{min e.v of }A|}
$$
We calculate the maximum and minimum eigenvalues of $A$ using the power and inverse power methods implemented in the Q-files. With this we get:
\begin{align*}
|\text{max e.v of }A| &= 2.0005 \\
|\text{min e.v of }A| &= 0.0005 \\
_\kappa(A) &= \frac{2.0005}{0.0005} \\
_\kappa(A) &= 4001
\end{align*}
Solving $Ax = b$ for $b = \begin{pmatrix} 2 \\ 2 \end{pmatrix}$ we get the vector:
$$
x =
\begin{pmatrix}
2 \\
0
\end{pmatrix}
$$
Solving $Ax = b$ for $b = \begin{pmatrix} 2 \\ 2.001 \end{pmatrix}$ we get the vector:
$$
x =
\begin{pmatrix}
1 \\
1
\end{pmatrix}
$$
This large change in the output makes sense, even with the extremely small change in the input, given our large condition number of $4001$.
